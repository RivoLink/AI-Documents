\documentclass{article}

\usepackage[utf8]{inputenc}
\usepackage{amsmath}
\usepackage{french}

\begin{document}

\vspace*{\fill}
\begingroup
\centering

Neural Network

\vspace{3mm}
Initiation aux réseaux de neurones artificiels

\vspace{5mm} 
rivo.link@gmail.com

\endgroup
\vspace*{\fill}

\newpage

\section{Neurone Biologique}

Comme tous les autres organes, le cerveau est composé de cellules. Les cellules du cerveau sont toutefois un peu spéciales, et ont des particularités. Ces cellules sont ce qu'on appelle des \textbf{neurones}.
\newline

Le neurone est composé d'un corps appelé \textbf{soma}, et de deux types de prolongements: \textbf{l'axone}, unique, et \textbf{les dendrites}, qui sont en moyenne 7 000 par neurone.
\newline

Un neurone reçoit des \textbf{entrées} ou signaux transmis par d’autres neurones. Au niveau du soma, le neurone analyse et traite ces signaux en les \textbf{sommant}. Si le résultat obtenu est supérieur au \textbf{seuil d'excitabilité}, il envoie une décharge alors nommé \textbf{potentiel d'action} le long de son axone vers d'autres neurones.
\newline

Le potentiel d'action obéit à la lois du \textbf{``Tout ou Rien''}, c'est à dire que, si le seuil d'excitabilité n'est pas atteint, le neurone n'émet rien, alors que si le seuil est dépassé, même de très peu, le potentiel d'action émis est de valeur maximal.

\section{Neurone Artificiel}























\end{document}
