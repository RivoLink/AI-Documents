\documentclass{article}

\usepackage{hyperref}

\usepackage{amsmath}
\usepackage[utf8]{inputenc}

\usepackage{listings}
\usepackage{color}

\definecolor{dkgreen}{rgb}{0,0.6,0}
\definecolor{gray}{rgb}{0.5,0.5,0.5}
\definecolor{mauve}{rgb}{0.58,0,0.82}

%default: \lstset{language=Java}
\lstset{
  frame=tb,
  language=Java,
  aboveskip=3mm,
  belowskip=3mm,
  showstringspaces=false,
  columns=flexible,
  basicstyle={\small\ttfamily},
  numbers=none,
  numberstyle=\tiny\color{gray},
  keywordstyle=\color{blue},
  commentstyle=\color{dkgreen},
  stringstyle=\color{mauve},
  breaklines=true,
  breakatwhitespace=true,
  tabsize=3
}

\begin{document}

\vspace*{\fill}
\begingroup
\centering
Backpropagation

Processus d'Apprentissage d'un Réseau de Neurones

rivo.link@gmail.com

\endgroup
\vspace*{\fill}

\newpage

\section{Backpropagation - Neurone}

\smallbreak
Ceci est mon approche de la backpropagation, un processus d'apprentissage d'un réseau de neurones multicouches.

\smallbreak
Soit un réseau de neurones à une couche cachée noté NN(i,j,k) où i est le nombre des entrées, j le nombre de neurones dans la couche cachée et k le nombre de neurones de sorties.

\smallbreak
On notera $(x_i)_i$ les entrées présentées au neurone j de la couche cachée, $(w_{ij})_i$ les poids associés et $b_j$ le biais pour ce neurone. Le neurone j aura donc la pré-activation $z_j$ selon la formule:
\begin{equation}
z_j=\sum_i{{x_i}{w_{ij}}}+b_j
\end{equation}

L'activation du neurone j de la couche cachée sera la sigmoïde de sa pré-activation, il aura donc comme sortie: 
\begin{equation}
f(z_j)=\frac{1}{1+e^{-z_j}}=f(\sum_i{{x_i}{w_{ij}}}+b_j)
\end{equation}

Comme les sorties des j neurones de la couche cachée sont les entrée des k neurones de la couche de sortie, en notant $(x_j)_j$ ces entrées, $(w_{jk})_j$ les poids et $b_k$ le biais, on aura pour le neurone k de la couche de sortie, la pré-activation $z_k$:
$$x_j=f(z_j)$$
\begin{equation}
z_k=\sum_j{x_j{w_{jk}}}+b_k
\end{equation}

L'activation du neurone k, de la couche de sortie, sera la sigmoïde de sa pré-activation, il aura donc comme sortie: 
\begin{equation}
f(z_k)=\frac{1}{1+e^{-z_k}}=f(\sum_j{x_j{w_{jk}}}+b_k)
\end{equation}

Finalement, en notant $\hat{y}_k$ la sortie du neurone k de la couche de sortie, alors $(\hat{y}_k)_k$ seront les sorties du réseau de neurones NN(i,j,k), et on aura les k-équations:
$$\hat{y}_k=f(z_k)$$
\begin{equation}
\hat{y}_k=f(\sum_j{f(\sum_i{x_i{w_{ij}}}+b_j){w_{jk}}}+b_k)
\end{equation}

\newpage

\section{Backpropagation - Apprentissage}

On note $(y_k)_k$ les sorties attendues, associée aux entrées $(x_i)_i$ pour le réseau de neurones NN(i,j,k). On définit l'erreur de prédiction $E(\hat{y})$ par:
\begin{equation}
E(\hat{y})=\sum_k{\frac{1}{2}(y_k-\hat{y}_k)^2}
\end{equation}

\subsection{Apprentissage - Couche de sortie}

La mise à jour du j-ème poids $w_{jk}$ du neurone k de la couche de sortie se fera selon la formule: 
\begin{equation}
w_{jk}:=w_{jk}-\alpha\frac{\partial{E(\hat{y})}}{\partial{w_{jk}}}
\end{equation}

\begin{equation}
\frac{\partial{E(\hat{y})}}{\partial{w_{jk}}}=\frac{\partial{}}{\partial{w_{jk}}}\sum_k{\frac{1}{2}(y_k-\hat{y}_k)^2}
\end{equation}

Comme, seul $\hat{y}_k$ dépend de $w_{jk}$, et les elements de la somme dont les indices se diffèrent de k ont une dérivée partielle nulle alors:
\begin{equation}
\begin{split}
\frac{\partial{E(\hat{y})}}{\partial{w_{jk}}}&=-\sum_k{(y_k-\hat{y}_k)\frac{\partial{\hat{y}_k}}{\partial{w_{jk}}}}\\
&=-(y_k-\hat{y}_k)\frac{\partial{\hat{y}_k}}{\partial{w_{jk}}}\\
&=-(y_k-\hat{y}_k)\frac{\partial{f(z_k)}}{\partial{w_{jk}}}\\
&=-(y_k-\hat{y}_k)\frac{\partial{f(z_k)}}{\partial{z_k}}\frac{\partial{z_k}}{\partial{w_{jk}}}\\
\frac{\partial{E(\hat{y})}}{\partial{w_{jk}}}
&=-(y_k-\hat{y}_k){f'(z_k)}\frac{\partial{}}{\partial{w_{jk}}}(\sum_j{{x_j}w_{jk}}+b_k)
\end{split}
\end{equation}

Finalement, comme "la fonction f est la fonction sigmoid", il en résulte l'équation de variation de l'erreur $E(\hat{y})$ par rapport au poid $w_{jk}$, et par rapport au biais $b_k$:
\begin{align*}
\frac{\partial{E(\hat{y})}}{\partial{w_{jk}}}&=-(y_k-\hat{y}_k)f(z_k)(1-f(z_k))x_j\\\frac{\partial{E(\hat{y})}}{\partial{b_k}}&=-(y_k-\hat{y}_k)f(z_k)(1-f(z_k))
\end{align*}

En d'autres termes, comme $\hat{y}_k=f(z_k)$, on a les nouvelles équations:

\begin{align*}\frac{\partial{E(\hat{y})}}{\partial{w_{jk}}}&=-\hat{y}_k(1-\hat{y}_k)(y_k-\hat{y}_k)x_j\\
\frac{\partial{E(\hat{y})}}{\partial{b_k}}&=-\hat{y}_k(1-\hat{y}_k)(y_k-\hat{y}_k)
\end{align*}

\newpage

\subsection{Apprentissage - Couche cachée}

La mise à jour du i-ème poids $w_{ij}$ du neurone j de la couche cachée se fera selon la formule: 
\begin{equation}
w_{ij}:=w_{ij}-\alpha\frac{\partial{E(\hat{y})}}{\partial{w_{ij}}}
\end{equation}

\begin{equation}
\frac{\partial{E(\hat{y})}}{\partial{w_{ik}}}=\frac{\partial{}}{\partial{w_{ij}}}\sum_k{\frac{1}{2}(y_k-\hat{y}_k)^2}
\end{equation}

Il faut noter qu'ici, la somme ne disparaît pas puisqu'elle ne depend pas de $w_{ij}$. Et comme seul $\hat{y}_k$ dépend de $w_{ij}$, alors on a:
\begin{align*}
\frac{\partial{E(\hat{y})}}{\partial{w_{ij}}}&=-\sum_k{(y_k-\hat{y}_k)\frac{\partial{\hat{y}_k}}{\partial{w_{ij}}}}\\
&=-\sum_k{(y_k-\hat{y}_k)\frac{\partial{f(z_k)}}{\partial{w_{ij}}}}\\
&=-\sum_k{(y_k-\hat{y}_k)\frac{\partial{f(z_k)}}{\partial{z_k}}\frac{\partial{z_k}}{\partial{x_j}}\frac{\partial{x_j}}{\partial{z_j}}\frac{\partial{z_j}}{\partial{w_{ij}}}}\\
&=-\sum_k{(y_k-\hat{y}_k){f'(z_k)}(\frac{\partial{}}{\partial{x_j}}\sum_j{{x_j}w_{jk}}+b_k)\frac{\partial{x_j}}{\partial{z_j}}\frac{\partial{z_j}}{\partial{w_{ij}}}}\\
&=-\sum_k{(y_k-\hat{y}_k){f'(z_k)}(w_{jk})\frac{\partial{x_j}}{\partial{z_j}}\frac{\partial{z_j}}{\partial{w_{ij}}}}\\
&=-\sum_k{(y_k-\hat{y}_k){f'(z_k)}(w_{jk})\frac{\partial{f(z_j)}}{\partial{z_j}}\frac{\partial{z_j}}{\partial{w_{ij}}}}\\
&=-\sum_k{(y_k-\hat{y}_k){f'(z_k)}(w_{jk}){f'(z_j)}\frac{\partial{z_j}}{\partial{w_{ij}}}}\\
\frac{\partial{E(\hat{y})}}{\partial{w_{ij}}}&=-\sum_k{(y_k-\hat{y}_k){f'(z_k)}(w_{jk}){f'(z_j)}(\frac{\partial{}}{\partial{w_{ij}}}}\sum_i{{x_i}w_{ij}}+b_j)
\end{align*}

Finalement, il en résulte l'équation de variation de l'erreur $E(\hat{y})$ par rapport au poid $w_{ij}$. Et on en déduit sa variation par rapport au biais $b_j$, toujours en notant que f est la fonction sigmoid:

\begin{align*}
\frac{\partial{E(\hat{y})}}{\partial{w_{ij}}}&=-\sum_k{(y_k-\hat{y}_k){f(z_k)(1-f(z_k))}w_{jk}{f(z_j)(1-f(z_j))}x_i}\\
\frac{\partial{E(\hat{y})}}{\partial{b_j}}&=-\sum_k{(y_k-\hat{y}_k){f(z_k)(1-f(z_k))}w_{jk}{f(z_j)(1-f(z_j))}}
\end{align*}

\newpage

En d'autres termes, toujours en utilisant les équations $\hat{y}_k=f(z_k)$ et $x_j=f(z_j)$, on a les nouvelles equations:

\begin{align*}
\frac{\partial{E(\hat{y})}}{\partial{w_{ij}}}&=-\sum_k{(y_k-\hat{y}_k){\hat{y}_k(1-\hat{y}_k)}w_{jk}{x_j(1-x_j)}x_i}\\
\frac{\partial{E(\hat{y})}}{\partial{b_j}}&=-\sum_k{(y_k-\hat{y}_k){\hat{y}_k(1-\hat{y}_k)}w_{jk}{x_j(1-x_j)}}
\end{align*}

\section{Backpropagation - Implémentation}

En informatique, la pratique est toujours plus explicite que la théorie, alors une implémentation en Java de la backpropagation sera donnée ci-après.
\smallbreak
La methode train de la classe Network est la principale implémentation de l'algorithme de backpropagation:

\begin{lstlisting}
public void train(float[] inputs,int[] target){
	float[] yhat=setInput(inputs).getOutputs();
	float[] x=hidL.getOutputs();
	int[] y=target;
		
	Neuron[] hN=hidL.neurons;
	Neuron[] oN=outL.neurons;
		
	float gE_wij,gE_wjk;
	for(int j=0;j<hN.length;j++){
		for(int k=0;k<oN.length;k++){
			gE_wjk=alpha*(y[k]-yhat[k])*yhat[k]*(1-yhat[k]);
			oN[k].biais+=gE_wjk;
			oN[k].weights[j]+=gE_wjk*x[j];
		}
	
		for(int i=0;i<this.i;i++){
			gE_wij=0;
			float xi=hidL.neurons[0].inputs[i];
			for(int k=0;k<oN.length;k++){
				float wjk=outL.neurons[k].weights[j];
				gE_wij+=(y[k]-yhat[k])*yhat[k]*(1-yhat[k])*wjk*x[j]*(1-x[j]);
			}
			gE_wij*=alpha;
			hN[j].biais+=gE_wij;
			hN[j].weights[i]+=gE_wij*xi;
		}
	}
}

\end{lstlisting}

\newpage

Ceci est la code source complete, il est aussi disponible sur: \url{https://github.com/RivoLink/AI_Doc/tree/master/backpropagation/java/src}

\begin{lstlisting}
// Neuron.java
import java.util.Random;

public class Neuron{

	static final Random r=new Random();
	
	public float biais;
	public final int size;
	public final float inputs[];
	public final float weights[];
	
	public Neuron(int size){
		this.size=size;
		this.inputs=new float[size];
		this.weights=new float[size];
		
		biais=r.nextFloat();
		for(int i=0;i<size;i++){
			weights[i]=r.nextFloat();
		}
	}
	
	public Neuron setInput(float inputs[]){
		for(int i=0;i<size;i++){
			this.inputs[i]=inputs[i];
		}
		return this;
	}
	
	public float getOutput(){
		float z=dot(inputs,weights)+biais;
		return (float)(1/(1+Math.exp(-z)));
	}

	public static float dot(float[] x,float[] w){
		if(x.length!=w.length)
			return 0;

		float dot=0;
		for(int i=0;i<x.length;i++){
			dot+=x[i]*w[i];
		}
		return dot;
	}
}

\end{lstlisting}

\newpage

\begin{lstlisting}
// Layer.java
public class Layer{
	
	public final int nCount;
	public final Neuron[] neurons;
	
	public Layer(int iLen,int nCount){
		this.nCount=nCount;
		
		neurons=new Neuron[nCount];
		for(int i=0;i<nCount;i++){
			neurons[i]=new Neuron(iLen);
		}
	}
	
	public Layer setInputs(float[] inputs){
		for(int i=0;i<nCount;i++){
			neurons[i].setInput(inputs);
		}
		return this;
	}
	
	public float[] getOutputs(){
		float[] outputs=new float[nCount];
		for(int i=0;i<nCount;i++){
			outputs[i]=neurons[i].getOutput();
		}
		return outputs;
	}
}

// Network.java
public class Network{
	public final float alpha=0.1f;
	
	public final int i;
	public final Layer hidL;
	public final Layer outL;
	
	public Network(int i,int j,int k){
		this.i=i;
		this.hidL=new Layer(i,j);
		this.outL=new Layer(j,k);
	}
	
	public Network setInput(float... bits){
		hidL.setInputs(bits);
		return this;
	}
	
	public float[] getOutputs(){
		float[] x=hidL.getOutputs();
		return outL.setInputs(x).getOutputs();
	}

	public void train(float[] inputs,int[] target){
		float[] yhat=setInput(input).getOutputs();
		float[] x=hidL.getOutputs();
		int[] y=target;
		
		Neuron[] hN=hidL.neurons;
		Neuron[] oN=outL.neurons;
		
		float gE_wij,gE_wjk;
		for(int j=0;j<hN.length;j++){
			for(int k=0;k<oN.length;k++){
				gE_wjk=alpha*(y[k]-yhat[k])*yhat[k]*(1-yhat[k]);
				oN[k].biais+=gE_wjk;
				oN[k].weights[j]+=gE_wjk*x[j];
			}
			
			for(int i=0;i<this.i;i++){
				gE_wij=0;
				float xi=hidL.neurons[0].inputs[i];
				for(int k=0;k<oN.length;k++){
					float wjk=outL.neurons[k].weights[j];
					gE_wij+=(y[k]-yhat[k])*yhat[k]*(1-yhat[k])*wjk*x[j]*(1-x[j]);
				}
				gE_wij*=alpha;
				hN[j].biais+=gE_wij;
				hN[j].weights[i]+=gE_wij*xi;
			}
		}
	}
}

\end{lstlisting}

\newpage

\section{Backpropagation - Apprentissage Profond}

En apprentissage profond, ou "Deep Learning", on ne se contente pas d'une ou deux couches cachées, il peut etre question de cent, cinq cent ou meme mille couches cachées, d'où vient l'appellation "profond".


\subsection{Apprentissage Profond - Les entrées}

En apprentissage profond, vu le nombre de couches cachées, la mise à jour des poids n'est plus chose aisée.
Supposons qu'on aie j,k,l,...,q couches cachées, les q-entrées $(x_q)_q)$ de la couche de sortie seront données par les q-equations suivantes:

$$z_q=\sum_p{f(... \sum_k{f(\sum_j{f(\sum_i{x_iw_{ij}}+b_j)w_{jk}}+b_k)w_{kl}}+b_l ...)w_{pq}}+b_q$$

En effet, 
\begin{align*}
z_j&=\sum_i{x_iw_{ij}}+b_j\\
z_k&=\sum_j{f(z_j)w_{jk}}+b_k\\
z_l&=\sum_k{f(z_k)w_{kl}}+b_l\\
...&=...\\
z_q&=\sum_p{f(z_p)w_{pq}}+b_q
\end{align*}

\subsection{Apprentissage Profond - Poids}
Ayant la variation de l'erreur de prediction $E(\hat{y})$ par rapport au poids $(w_{ij})=)_i$ des j-neurones de la premiere couche cachée, toutes les autres variation s'en decoulent, alors:

\smallbreak

Avec,
$$x_k=f(z_k), z_k=\sum_j{f(z_j)w_{jk}}+b_k$$

On a:
\begin{align*}
\frac{\partial{E(\hat{y})}}{\partial{w_{ij}}}&=-\sum_q{(y_k-\hat{y}_q)}{f'(z_q)}{w_q}{f'(z_p)}...{f'(z_k)}{w_k}{f'(z_j)}{w_j}{f'(z_i)}x_i\\
\frac{\partial{E(\hat{y})}}{\partial{b_i}}&=-\sum_q{(y_k-\hat{y}_q)}{f'(z_q)}{w_q}{f'(z_p)}...{f'(z_k)}{w_k}{f'(z_j)}{w_j}{f'(z_i)}
\end{align*}

\newpage

En effet,

\begin{align*}
\frac{\partial{E(\hat{y})}}{\partial{w_{ij}}}&=-\sum_q{(y_q-\hat{y}_q)\frac{\partial{\hat{y}_q}}{\partial{w_{ij}}}}\\
&=-\sum_q{(y_k-\hat{y}_q)\frac{\partial{f(z_q)}}{\partial{w_{ij}}}}\\
\frac{\partial{E(\hat{y})}}{\partial{w_{ij}}}&=-\sum_q{(y_q-\hat{y}_q)\frac{\partial{f(z_q)}}{\partial{z_q}}\frac{\partial{z_q}}{\partial{x_p}}\frac{\partial{x_p}}{\partial{z_p}}...\frac{\partial{x_k}}{\partial{z_k}}\frac{\partial{z_k}}{\partial{x_j}}\frac{\partial{x_j}}{\partial{z_j}}\frac{\partial{z_j}}{\partial{w_{ij}}}}
\end{align*}

\end{document}
