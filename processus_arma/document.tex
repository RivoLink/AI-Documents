\documentclass{article}

\usepackage[utf8]{inputenc}
\usepackage{amsthm}
\usepackage{amsmath}

\usepackage{french}

\newtheorem*{asavoir}{A savoir}
\newtheorem{theorem}{Theorem}[section]
\newtheorem{intro}{Introduction}[section]
\newtheorem{definition}{Definition}[section]

\title{Série Temporelle - De la théorie à la pratique.}

\author{ANDRIAMANANA H. Rivo Hery}

\begin{document}

\maketitle

\section{Serie Temporelle - La Théorie}

\subsection{Auto-Regression (AR)}

\begin{definition}
Un processus $(X_t)$ est Auto-Regressif quand sa valeur à l'instant t n'est expliquée que par ses anciennes valeurs $(X_{t-1},...,X_{t-i})$, où $i\in\{2,...,\infty\}$ et non par d'autres processus.
$$X_t=\phi_1{X_{t-1}}+...+\phi_i{X_{t-i}}+\epsilon_t, \quad i\in\{2,...,\infty\},$$
où les $\epsilon_t$ sont des bruits blancs, indépendants et identiquement distribués, notés $\epsilon_t \rightarrow iid(0,\sigma^2), \; \forall t$.
\end{definition}

\begin{definition}[Ordre d'un AR]
Un processus AR est d'ordre p, noté AR(p), quand sa valeur à l'instant t est expliquée par ses p anciennes valeurs: 
$$X_t=\phi_1{X_{t-1}}+...+\phi_p{X_{t-p}}+\epsilon_t,$$
où $\epsilon_t \rightarrow iid(0,\sigma^2), \; \forall t$.
\end{definition}

\begin{definition}[Opérateur de retard]
Pour une série temporelle $(X_t)_t$, on définit l'opérateur de retard, noté L, par une application qui à chaque élément $X_t$ de la série, associe son observation précédente $X_{t-1}$:
$$LX_t=X_{t-1}, \quad \forall t>1.$$
En particulier, $L^i(X_t)=X_{t-i}$.
\end{definition}

\begin{definition}[Opérateur de différenciation]
Pour une série temporelle $(X_t)_t$, on définit l'opérateur de différenciation, noté $\nabla$, par une application qui à chaque élément $X_t$ de la série, associe la différence $X_t-X_{t-1}$:
$$\nabla{X_t}=X_t-X_{t-1}, \quad \forall t>1.$$
\end{definition}

\begin{definition}[Polynome caractéristique]
Ayant définit l'opérateur de retard, on peut l'utiliser dans la définition d'un processus AR(p).
\newline
Ainsi, si $(X_t)_t$ est un AR(p), alors, on définit le polynome caractéristique $\Phi$ d'un processus AR(p) de tel sorte que: $\Phi(L){X_t}=\epsilon_t$,
$$\Phi(L)=1-\phi_1{L^1}+...+\phi_p{L^p}.$$
\end{definition}

\begin{definition}[Équation caractéristique]
On appelle équation caracteristique d'un processus AR(p), l'équation déduit de $\Phi$ en remplacant L par x:
$$(1-\phi_1{x^1}+...+\phi_p{x^p}).$$\end{definition}

\subsection{Moyenne Mobile (MA)}

\begin{intro}
Une moyenne est dite mobile lorsqu'elle est recalculée de façon continue, en utilisant à chaque calcul un sous-ensemble d'éléments dans lequel un nouvel élément remplace le plus ancien ou s'ajoute au sous-ensemble. 
\end{intro}

\begin{definition}
Un processus est une Moyenne Mobile lorsqu'il est de la forme:
$$X_t=\theta_1{\epsilon_{t-1}}+...+\theta_i{\epsilon_{t-i}}+\epsilon_t, \quad i\in\{2,...,\infty\},$$
où $\epsilon_t \rightarrow iid(0,\sigma^2), \; \forall t$.
\end{definition}

\begin{definition}[Ordre d'un MA]
Un processus MA est d'ordre q, noté MA(q), quand sa valeur à l'instant t est expliquée par ses q anciennes valeurs: 
$$X_t=\theta_1{\epsilon_{t-1}}+...+\theta_p{\epsilon_{t-q}}+\epsilon_t,$$
où $\epsilon_t \rightarrow iid(0,\sigma^2), \; \forall t$.
\end{definition}

\begin{definition}[Polynome caractéristique]
Le polynome caractéristique $\Theta$ d'un processus MA(q) est definit de tel sorte que: $X_t=\Theta(L)\epsilon_t$,
$$\Theta(L)=1+\theta_1{L^1}+...+\theta_q{L^q}.$$
\end{definition}

\subsection{Auto-Regression et Moyenne Mobile (ARMA)}

\begin{definition}
Un processus ARMA $(X_t)_t$ est comme son nom l'indique, un processus auto-regessif et moyenne mobile. Il a une partie AR(p) et une partie MA(q) et est noté ARMA(p,q) selon la définition:
$$X_t:=\phi_1{X_{t-1}}+...+\phi_p{X_{t-p}}+\theta_1{\epsilon_{t-1}}+...+\theta_p{\epsilon_{t-q}}+\epsilon_t,$$
où $\epsilon_t \rightarrow iid(0,\sigma^2), \; \forall t$.
\end{definition}

\begin{definition}[Stationnarité faible]
Un processus $(X_t)_t$ est faiblement stationnaire si: $\forall t$, $\forall h<t$;
\begin{itemize}
\item $E(X_t)=\mu$, l'esperence est constante au cours du temps.
\item $Var(X_t)=\sigma^2<\infty$, la variance est constante et non infinie.
\item $Cov(X_t,X_{t-h})=\gamma(h)$, l'auto-corrélation entre $X_t$ et $X_{t-h}$ reste constante et ne dépend que de h.
\end{itemize}
\end{definition}

\begin{definition}[Stationnarité forte]
Un processus $(X_t)_t$ est fortement stationnaire si: $\forall t$, $\forall h$; $(X_1,X_2,...,X_t)$ et $(X_{1+h},X_{2+h},...,X_{t+h})$ ont même lois en probabilité.
\end{definition}

\begin{definition}[Auto-Corrélation (ACF)]
La fonction d'auto-corrélation est définie par:
$$\rho(h)=\frac{\gamma(h)}{\gamma(0)}=\frac{E[(X_t-\mu)(X_{t+h}-\mu)]}{\sigma^2}.$$


\end{definition}

\newpage

\section{Serie Temporelle - La Pratique}

\begin{intro}
En analyse théorique, l'étude d'une série temporelle commence par sa forme théorique. Par exemple, l'étude d'un processus ARMA(p,q) $(X_t)_t$ débute par sa forme:

$$X_t=\phi_1{X_{t-1}}+...+\phi_p{X_{t-p}}+\theta_1{\epsilon_{t-1}}+...+\theta_p{\epsilon_{t-q}}+\epsilon_t,$$
où les $\phi_i$ et $\theta_j$ sont sont des constantes données $\forall i<p, \; \forall j<q$.
\newline

Par contre, l'analyse pratique d'une série temporelle débute par un tableau de données, et grâce à ces données, on essaie d'estimer les paramètres du processus, et de déterminer les propriétés des résidus.

\end{intro}
\begin{asavoir}[p-value]
En test statistique, le p-value est une critère d'acceptation de l'hypothèse nulle $(H_0)$:
\begin{itemize}
\item si $p>0.10$, aucune signification, $(H_0)$ acceptee
\item si $p\leq0.10$, asymptotiquement significative, $(H_0)$ acceptee
\item si $p\leq0.05$, significative, $(H_0)$ rejetee
\item si $p\leq0.01$, tres significative, $(H_0)$ rejetee.
\end{itemize}
\end{asavoir}

\begin{asavoir}[Teste de Dickey-Fuller Augmenté - ADF]
Le test ADF est un test de racine unitaire:
\newline
$(H_0):$ Le processus admet une racine unitaire (non stationnaire)
\newline
$(H_1):$ Le processus est stationnaire.
\newline
Package: library('tseries')
\newline
Utilisation: adf.test(...)
\end{asavoir}

\end{document}
